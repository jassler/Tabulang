\section{Import Export}
\label{sectionImportExport}

Mit Hilfe dieser Implementierung ist es möglich, einen Import- bzw. Exportprozess zu starten. Diese beiden Funktionalitäten stehen für das Arbeiten mit der relationalen Datenbank \textbf{MySQL} und der Open Office Lösung \textbf{Libre Office} und dem dazugehörigen Kalkulationsprogramm \textbf{Calc} bereit. Es ist somit möglich, Daten aus einer Tabellensprache in einer MySQL-Datenbank zu archivieren und auszulesen. Außerdem können Daten aus einer *.ods-Datei, die das Dateiformat von Libre Office Calc darstellt, ausgelesen werden. Ferner kann aus der Tabellensprache eine *.ods-Datei generiert werden. In den nächsten zwei Unterkapiteln \ref{subsectionMySql} und \ref{subsectionLibreOffice} werden die einzelnen Kernfunktionalitäten näher erläutert.

\subsection{MySQL}
\label{subsectionMySql}

\subsubsection{Models}
\label{subsubsectionMySqlModels}
\textbf{MSqlConnectionParameters}

\begin{lstlisting}[label={labelMySqlConnectionParameters},caption=Parameter für eine MySQL-Datenbankverbindung, language=Java]
private String _ip;
private int _port;
private String _dbName;
private String _username;
private String _password;
\end{lstlisting}

\textbf{MSqlTableContent}

\begin{lstlisting}[label={labelMySqlConnectionParameters},caption=Repräsentation einer MySQL-Tabelle als Java-Objekt, language=Java]
private String _dbName;
private ArrayList<String> _headlines;
private ArrayList<ArrayList<String>> _content;
\end{lstlisting}

\subsubsection{Öffentliche Funktionen}
\label{subsubsectionMySqlPublicFunctions}

\subsubsection{Private Funktionen}
\label{subsubsectionMySqlPrivateFunctions}

\subsection{Libre Office}
\label{subsectionLibreOffice}

\subsubsection{Models}
\label{subsubsectionLibreOfficeModels}

\textbf{MCell}

\begin{lstlisting}[label={labelMySqlConnectionParameters},caption=Repräsentation einer Zelle einer Libre Office Calc Datei, language=Java]
private HashMap<String,String> Attributes;
private Object _value;
\end{lstlisting}

\textbf{MColumn}

\begin{lstlisting}[label={labelMySqlConnectionParameters},caption=Repräsentation einer Spalte einer Libre Office Calc Datei, language=Java]
private HashMap<String,String> Attributes;
\end{lstlisting}

\textbf{MRow}

\begin{lstlisting}[label={labelMySqlConnectionParameters},caption=Repräsentation einer Zeile einer Libre Office Calc Datei, language=Java]
private HashMap<String,String> Attributes;
private ArrayList<Cell> _cells;
\end{lstlisting}

\textbf{MSpreadsheet}

\begin{lstlisting}[label={labelMySqlConnectionParameters},caption=Repräsentation einer kompletten Libre Office Calc Datei, language=Java]
private ArrayList<HashMap<String, String>> _fontStyles;
private HashMap<String, HashMap<String, String>> _tableStyles;
private HashMap<String, HashMap<String, String>> _rowStyles;
private HashMap<String, HashMap<String, String>> _columnStyles;
private HashMap<String, HashMap<String, String>> _cellStyles;
private ArrayList<TableWrapper> _tables;
\end{lstlisting}

\textbf{MStyle}

\begin{lstlisting}[label={labelMySqlConnectionParameters},caption=Repräsentation eines Styles einer Libre Office Calc Datei, language=Java]
private OdfStyleProperty property;
private String value;
\end{lstlisting}

\textbf{MTableWrapper}

\begin{lstlisting}[label={labelMySqlConnectionParameters},caption=Repräsentation einer Tabelle einer Libre Office Calc Datei inkl. aller Zeilen und Spalten, language=Java]
private HashMap<String,String> _attributes;
private ArrayList<Column> _columns;
private ArrayList<Row> _rows;
\end{lstlisting}

\subsubsection{Öffentliche Funktionen}
\label{subsubsectionLibreOfficePublicFunctions}

\subsubsection{Private Funktionen}
\label{subsubsectionLibreOfficePrivateFunctions}